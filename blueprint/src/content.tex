\title{Abstract Algebra in Lean}


%\home{https://XintaoYu.github.io/blueprint-template/}
%\github{https://github.com/XintaoYu/Abstract-Algebra-in-Lean}
%\dochome{https://reaslab.github.io/blueprint-template/doc/}

% \home{localhost:8080}
% \dochome{localhost:8080/doc}

\maketitle


\tableofcontents
\section{Introduction}



\section{Exercise}

% To avoid bibtex errors
\nocite{*} % Delete this line if you have citations.

\begin{theorem}[Exercise 1]\label{Ex1}
  \leanok
    Suppose that $*$ is an associative binary operation on a set $S$. Let
    $$
    H = \{a\in S | a * x = x * a, \forall x\in S\}
    $$

    Show that $H$ is under $*$. (We think of $H$ as consisting of all elements of $S$ that commute with every element in $S$.)
\end{theorem}
\begin{proof}
  \leanok
\end{proof}


\begin{theorem}[Exercise 2]\label{Ex2}
  \leanok
    Are all groups of 4 elements commutative?
\end{theorem}
\begin{proof}
  \leanok
\end{proof}


\begin{theorem}[Exercise 3]\label{Ex3}
  \leanok
    Show that if $G$ is a finite group with identity $e$ and with an even number of elements, then there is $a \ne e$ in $G$ such that $a * a = e$.
\end{theorem}
\begin{proof}
  \leanok
\end{proof}


\begin{theorem}[Exercise 4]\label{Ex4}
  \leanok
    Show that if $H$ and $K$ are subgroups of an abelian group $G$, then
    $$
    \{hk|h \in H \and k \in K\}
    $$
    is a subgroup of G.
\end{theorem}
\begin{proof}
  \leanok
\end{proof}

\begin{theorem}[Exercise 5]\label{Ex5}
  \leanok
Show that a group with no proper nontrivial subgroups is cyclic.

\end{theorem}
\begin{proof}
  \leanok
\end{proof}

\begin{theorem}[Exercise 6]\label{Ex6}
  \leanok
    Let $p$ be a prime number. Find the number of generators of the cyclic group $Z_{p^r}$ , where $r$ is an integer $\ge$ 1.
\end{theorem}
\begin{proof}
  \leanok
\end{proof}

\begin{theorem}[Exercise 13]\label{Ex13}
  \leanok
    Let $G$ be a group and let $a$ be fixed element of $G$. Show that the map $\lambda_a: G\to G$, given by $\lambda_a(g)=ag$ for $g\in G$, is a permutation of the set $G$.
\end{theorem}
\begin{proof}
  \leanok
\end{proof}